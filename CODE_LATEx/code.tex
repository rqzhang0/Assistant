\documentclass{article}
\usepackage[english]{babel}
\usepackage{amsmath}

\begin{document}

\title{CODES' NOTE}

\maketitle

FUCKING CODER

\section{SHELL }

\subsection{Basic Function}

\

The Function of SHELL:

Interpret the user's input and excute these commands.

\

Shell do three things in its {\bfseries{LifeTime:}}
\begin{itemize}
  \item Initialize
  
  \item Interpret
  
  \item Terminate
\end{itemize}


\

The functions of shell can be divided into three parts:
\begin{itemize}
  \item Read Command
  
  \item Understand Command
  
  \item Excute Command
\end{itemize}


{\bfseries{Read Command:}}

Generally, Shell read in a string and store it into an array.

{\bfseries{step1.}} To read in Command, we need \ {\bfseries{getchar()}}
function.

\ \ \ \ \ \ \ \ \ \ {\bfseries{getchar()}} function read in one single char
each time. So we need a loop to make it.

{\bfseries{step2.}} We need to decide when to finish reading.

\ \ \ \ \ \ \ \ \ \ Typically when we accept an {\bfseries{Enter}} or
{\bfseries{EOF}}, we need to stop reading in.

\ \ \ \ \ \ \ \ \ \ \ {\bfseries{EOF:}} end of file. value is $- 1$.

\

{\bfseries{Understand Command:}}

It means that we need to extract information from strings. Typically, we need
to know what program is required, what arguments it has. To simplify our
program, we understand the input as

{\bfseries{Programm+argumenmts.}}

It means that the first char should be the name of the program we want to
call. The left chars are these required arguments.

These commands and arguments should be divided by space, tab($\backslash
t${\bfseries{}}),return(r),newline(n),vertical tab(v).

Namely we have

{\bfseries{step1.}} split string according to there characters. we use
{\bfseries{strtok}} function to split string. And we also need a loop to
iteratively split the string.

{\bfseries{step2.}} store there char into array.

\

{\bfseries{Excuted Commands:}}

\

\section{My First Own Project: {\bfseries{Assistant}}}

\

{\bfseries{The advantage of this CODE:}}
\begin{itemize}
  \item If your code is correct, you don't have to check your code every time;
  
  \item It would work for different work;
  
  \item It would work for different operators, you don't have to wirte code
  explicitely for each operator. Namely, if you want you can easily generate
  the code that calculate 12 different kinds of \ operators. 
\end{itemize}
It's function would include
\begin{itemize}
  \item code generation (c);
  
  \item File management, including encryption, sorting, searching ... and
  writting assistant;
  
  \item Reminder;
  
  \item Help derivation;
  
  \item Automatic web searching, sorting, saving, analyzing...
\end{itemize}


This project would be a highly personalize and it will be an assisant of
myself.

\subsection{Assistant for Contraction}


\begin{eqnarray*}
  C (t) & = & \Phi_{\alpha \beta}^{B a b} (t') \tau_{\beta \gamma}^{b c} (t',
  t) \Phi_{\gamma \kappa}^{A c d} (t) \tau_{\kappa \alpha}^{d a} (t, t')
\end{eqnarray*}
Conver this line (latex source code) to

{\bfseries{Pseudo code:}}

for a,b,c,d

for $\alpha, \beta, \gamma, \kappa$

for($t'$)

$\begin{array}{lll}
  C (t) & = & \Phi_{\alpha \beta}^{B a b} (t') \ast \tau_{\beta \gamma}^{b c}
  (t', t) \ast \Phi_{\gamma \kappa}^{A c d} (t) \ast \tau_{\kappa \alpha}^{d
  a} (t, t')
\end{array}$

$\Rightarrow$

corr[0][t\_src][t]+=vector\_products[0][t\_snk][c1][c2]*gamma[s1][s2]*peram[t\_src][t][s2][s3][c2][c3]*conj(vector\_products[0][t\_src][c4][c3])*gamma[s3][s4]*peram\_back[t\_src][t][s4][s1][c4][c1];

\

\

{\bfseries{Convert function:}}
\begin{itemize}
  \item {\bfseries{Step1.}}split line according to *, and $=$ ;
  
  \item {\bfseries{Step2}}.split token according to bracket ']' and '[';
  
  \item {\bfseries{Step3.}}split these string in $[\ldots]$ according to
  $','$and interpret it as index ;
  
  \item {\bfseries{Step4.}}Determine repeated indexes;
  
  \item {\bfseries{Step5.}}Write a {\bfseries{for}} loop for each repeated
  index.
\end{itemize}
{\bfseries{Qusetions:}}
\begin{itemize}
  \item {\bfseries{Q1.}}Firstly, How to determine which index is repeated;
  
  \item {\bfseries{Q2.}} How to combine these variables with corresponding
  indexes.
  
  \item {\bfseries{Q3.}} How to deal with these indexes that repreated but
  doesn't need loop over ?
  
  \item {\bfseries{Q4.}} How to deal with these operators 
\end{itemize}
{\bfseries{Solutions:}}
\begin{itemize}
  \item {\bfseries{S1}}. \ We should write a simple function to search
  repeated indexes;
  
  \item {\bfseries{S2.}} \ We should sotore variables and correspondings as
  $\ensuremath{\operatorname{var}} [i] [j]$. $\ensuremath{\operatorname{var}}
  [i] [0]$ stores the variable name, $\ensuremath{\operatorname{var}} [i] [1 -
  n]$ store the indexes of i-th variable;
  
  \item {\bfseries{S3.}} \ \ $\Phi$ has only one time index while $\tau$ has
  two time indexes. But we don't have to repeated with these time indexes, so
  we don't have to deal with it. $\Phi$ looks like $\Phi [p, t, \ldots]$,
  $\tau$ looks like $\tau [t, t', \ldots]$ So we just only need to left first
  two indexes unchanged.
  
  \item {\bfseries{S4.}} For simplicity, we don't deal with bracket operaters
  seperately. Our input only include operators like $+, -, =, \ast \cdot$
  These operaters should be intepreted in a sequence. Firstly, the $' ='$.
  Secondly, $' +'$ and $' -'$. Finally, the $' \ast'$. In this way, we can
  know what operator is intepreting. To not forget any operator, we should
  store operators with correspoinding variables. For example, $a = b + c$. We
  should interpret it as $a$, $= b$, $+ c$. Genearlly, for a expression like
  $a = b \ast f + c - g$ \ \
  \begin{itemize}
    \item {\bfseries{Step1.}} Interpret it as $a$, $= b \ast f$, $+ c$, $- g$;
    
    \item {\bfseries{Step2.}} Interpret $= b \ast f$ as $= b$, $\ast f$;
    
    \item {\bfseries{Step3.}} combine $a, = b, \ast f, + c, - g$ to be c code;
    $a = b \ast f + c - g$. We only need to just combine these operators.
    
    \item {\bfseries{Step4.}} Usually, an variable would have several indexes
    so we need to interpret these indexes. For example, $a [t, t', \alpha,
    \beta, a, b]$ should be interpret as $a [t] [t'] [\alpha] [\beta] [a]
    [b]$. It should be split by $[ $ and $','$.
  \end{itemize}
\end{itemize}


\end{document}
